\section{Introduction}

Full multigrid (FMG) is a provably asymptotically exact, non-iterative algebraic equation solver for discretized
elliptic partial differential equations with work complexity of about five residual calculations, or what is known as
textbook multigrid efficiency (TME), for the constant coefficient Laplacian~\cite{BrandtDiskin1994}. For some classes of
elliptic problems, TME can be shown analytically, but it has been observed experimentally for a very broad spectrum of
problems~\cite{ThomasDiskinBrandt2001,trottenberg2001multigrid,Adams-10a}. In addition, the Full Approximation Scheme
(FAS) multigrid is applicable to general nonlinear elliptic equations.

Multigrid methods are widely used in practice, but distributed memory parallel multigrid presents many implementation
challenges. The benefits for future computer architectures of the reduction in communication, measured in both power and
time, are well-documented in~\cite{AdamsBrownKnepleySamtaney2016}. Segmental refinement explicitly decouples subdomain
processing on the finest multigrid levels, which improves data locality, amortizes latency costs, and reduces data
dependencies. However, large surveys of the scientific computing userbase, such as~\cite{PrabhuEtAl2011}, show that a
majority of practitioners are confined to a desktop computer. In addition, those users with parallel code have very long
running jobs (${} > 50$\% run for more than 24 hours), and make very little attempt to optimize the code, leading to the
conclusion that the leading reason for parallelism is to run jobs too large for a single machine. Our limited memory
implementation of SRMG can allow practitioners to run very large potential problems ($10^{12}$ unknowns) in the same 24
hour timeframe on a standard desktop machine. This capability has the potential to revolutionize scientific computing
for the vast mass of users.

\section{Background}

Beginning in the 1970s, Brandt~\cite{Brandt77,DinarThesis1979,brandt1984} proposed segmental refinement multigrid (SRMG)
as a polylogarithmic memory alternative to traditional FMG that does not store the entire solution in memory at any one
time. More recent work~\cite{BrandtDiskin1994,MohrRude1998,Mohr2000,AdamsBrownKnepleySamtaney2016} has concentrated on
the benefits for distributed memory computing, namely that SRMG has no interprocess communication on the finest
grids. This paper presents an implementation with polylogarithmic memory requirements, and experimentally examines the
accuracy, memory scalability, and dependence on interpolation order. We present multilevel numerical results,
complementing~\cite{AdamsBrownKnepleySamtaney2016}, and verify complexity models for both computation and memory.

\section{Methodology}

We follow~\cite{BrandtDiskin1994} in employing a vertex-centered discretization for the multigrid iteration, in this
case a 5-point finite difference stencil on a Cartesian grid.

\subsection{Computational and Memory Models}

\section{Results}

List all plots and the reason for each plot.

\section{Conclusions}
